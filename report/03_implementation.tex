\section{Implementation}

The implementation of the FPL optimization system consisted of five major phases: (1) Data Integration, (2) Data Cleaning, (3) MILP Model Implementation, (4) CP MiniZinc Model Implementation, and (5) Backtesting and Validation. This section details the technical implementation of each phase, highlighting key challenges, solutions, and noteworthy design decisions.

\subsection{Implementation Overview}

The system was implemented primarily in Python 3.10+, leveraging several key technologies and libraries:

\begin{itemize}
    \item \textbf{Data Processing}: pandas, numpy, csv for data manipulation and integration
    \item \textbf{Optimization Solvers}: 
        \begin{itemize}
            \item MILP: PuLP library with CBC, GLPK, and SCIP solvers
            \item CP: MiniZinc with CP-SAT, Chuffed and Gecode
        \end{itemize}
    \item \textbf{Development}: Jupyter notebooks for exploratory work, Python scripts for production code
\end{itemize}

\subsection{Phase 1: Data Integration}

High-quality input data forms the foundation of any data-driven optimization system. A primary challenge for this Fantasy Premier League (FPL) team management system was the lack of a single open dataset providing complete time-series player buying prices across the 2024-25 season. FPL prices fluctuate daily by up to £0.1 million due to transfer demand, but available datasets offer only gameweek snapshots without intra-week timing details.

Three complementary open-source sources were integrated for cross-validation. The OLB Dataset supplies player statistics, prices (\texttt{now\_cost}), status, and expected points. The VAA Dataset delivers detailed performance metrics, including minutes played and starting appearances. Randdalf's fplcache caches official FPL API data as a third reference.

This process spanned Gameweeks 1 to 38, producing a unified dataset with 27,222 unique (player, gameweek) records. The subsequent discussion outlines the technical challenges encountered and solutions implemented.


\subsubsection{OLB Dataset Processing}

The OLB dataset comprised two primary files: \texttt{players.csv} for player metadata and \texttt{playerstats.csv} for per-gameweek statistics. Processing entailed three essential steps.

\paragraph{Step 1: ID-to-Name Mapping}

The initial structural challenge arose from \texttt{playerstats.csv} using numeric player IDs, incompatible with the VAA dataset's player names. Mapping required a validation-first strategy to link IDs to full names.

The following implementation was employed:
\begin{lstlisting}[language=Python, caption={Player ID Mapping with Validation}]
# Create id-to-name mapping
players_df['full_name'] = (players_df['first_name'] + ' ' + 
                            players_df['second_name'])
id_to_name = dict(zip(players_df['player_id'], 
                      players_df['full_name']))

# Validate: ensure all playerstats IDs exist in players.csv
player_ids = set(players_df['player_id'].unique())
playerstats_ids = set(playerstats_df['id'].unique())

if player_ids >= playerstats_ids:  # Superset check
    playerstats_df['player_name'] = playerstats_df['id'].map(id_to_name)
    mapped_df = playerstats_df.drop('id', axis=1)
\end{lstlisting}

Superset verification confirmed that all 631 distinct IDs in \texttt{playerstats.csv} matched entries in \texttt{players.csv}, preserving integrity for merging.

\paragraph{Step 2: Column Filtering}

The mapped dataset included extraneous variables, thus only those vital to the optimization model were retained: \texttt{player\_name}, \texttt{status}, \texttt{chance\_of\_playing\_next\_round}, \texttt{chance\_of\_playing\_this\_round}, \texttt{now\_cost}, \texttt{event\_points} (gameweek-specific FPL points), \texttt{ep\_next} and \texttt{ep\_this} (expected points), and \texttt{gw}.

A key insight emerged regarding \texttt{total\_points}: in the OLB dataset, it denotes cumulative points from GW1, contrasting with the VAA dataset's gameweek-specific scoring. This semantic distinction informed later validations.

\paragraph{Step 3: Duplicate Verification}

Prior to integration, data quality was assessed by scanning for duplicate (gameweek, player name) pairs, anomalies given the standard one-match-per-club schedule.

Validation revealed zero duplicates across the OLB dataset's 27,658 records, yielding a reliable foundation. This cleanliness facilitated resolution of duplicates later identified in the VAA dataset.

\subsubsection{VAA Dataset Processing}

The VAA dataset introduced greater complexities, comprising 38 separate CSV files (\texttt{gw1.csv} through \texttt{gw38.csv}), one per gameweek.

\paragraph{Step 1: Column Inconsistency Detection}

Initial merge attempts faltered owing to mid-season structural alterations. Files \texttt{gw22.csv} through \texttt{gw38.csv} incorporated seven extraneous columns absent from earlier files:

\begin{verbatim}
Extra columns: mng_clean_sheets, mng_draw, mng_goals_scored,
               mng_loss, mng_underdog_draw, mng_underdog_win, 
               mng_win
\end{verbatim}

This schema evolution likely arose from upstream source modifications. The resolution employed conditional ingestion: direct reading for GW1--21; for GW22--38, exclusion of specified columns via index-based filtering to restore uniformity.

\paragraph{Step 2: Validated File Merging}

All 38 files were merged with enforced column alignment. Early files (GW1--21) underwent standard ingestion; later files (GW22--38) first confirmed structural parity post-exclusion, then proceeded with filtering. A gameweek identifier column was appended to each row for temporal fidelity. This yielded 27,605 rows of consistent data across the season.

\paragraph{Step 3: Column Filtering}

Analogous to OLB processing, the merged VAA data was pruned to indispensable fields: \texttt{name}, \texttt{position}, \texttt{team}, \texttt{xP} (expected points), \texttt{minutes}, \texttt{starts} (starting lineup indicator), \texttt{total\_points} (gameweek-specific points), \texttt{value} (player price), and \texttt{gw}.

\paragraph{Step 4: Duplicate Detection and Analysis}

In contrast to the OLB dataset, duplicate scrutiny identified 374 (gameweek, name) pairs, clustered in GW24 (112 pairs), GW25 (98 pairs), GW32 (82 pairs), and GW33 (82 pairs). Each instance duplicated precisely twice, with divergent content per row. Examination of the raw data pinpointed the cause: fixture rescheduling from weather or other factors prompted double matches for select clubs, as indicated by disparate kickoff times and opponent codes.


\subsubsection{Cross-Dataset Integration and Validation}

Following independent processing of both datasets, merging required careful handling of duplicate fixture records.

\paragraph{Step 1: Hypothesis Testing for Duplicate Validation}

Prior to aggregation, consistency was tested for duplicated players: the summed \texttt{total\_points} from VAA's two matches should equal OLB's \texttt{event\_points}.

\begin{lstlisting}[language=Python, caption={Cross-Dataset Validation}]
# Sum total_points for duplicated pairs
vaa_summed = duplicated_df.groupby(['gw', 'name'])['total_points']
                          .sum().reset_index()

# Merge with OLB event_points
merged = vaa_summed.merge(olb_df[['gw', 'player_name', 'event_points']],
                          left_on=['gw', 'name'],
                          right_on=['gw', 'player_name'])

# Validate: VAA sum == OLB event_points
num_matches = (merged['vaa_sum_total_points'] == 
               merged['event_points']).sum()
\end{lstlisting}

This check succeeded for all 374 pairs, affirming cross-source reliability and duplicate semantics.

\paragraph{Step 2: Duplicate Aggregation}

With validity established, aggregation rules preserved key information while eliminating duplicates. Additive metrics (\texttt{xP}, \texttt{total\_points}) were summed across matches. The \texttt{starts} field took the maximum to flag any starting appearance. Invariant attributes (\texttt{name}, \texttt{position}, \texttt{team}) retained their first value. For \texttt{value} (price) and \texttt{minutes}, averages were computed and rounded (to nearest £0.1M and integer). Price averaging accounts for minor intra-week variance without overcomplicating FPL mechanics; minutes averaging prioritizes simplicity over fixture-specific modeling. This condensed the VAA dataset to 27,231 unique rows.

\paragraph{Step 3: Final Duplicate Verification}

Pre-integration checks confirmed uniqueness:

\begin{lstlisting}[language=Python]
# Verify uniqueness in both datasets
vaa_unique = len(vaa_df.groupby(['gw', 'name'])) == len(vaa_df)
olb_unique = len(olb_df.groupby(['gw', 'player_name'])) == len(olb_df)
# Both returned True
\end{lstlisting}

\paragraph{Step 4: Dataset Matching Analysis}

Overlap analysis of (player, gameweek) pairs revealed:

\begin{table}[h]
\centering
\begin{tabular}{|c|c|c|c|c|c|}
\hline
\textbf{OLB Pairs} & \textbf{VAA Pairs} & \textbf{Matching Pairs} & \textbf{Match \%} & \textbf{Only in OLB} & \textbf{Only in VAA} \\
\hline
27,658 & 27,231 & 27,222 & 98.43\% & 436 & 9 \\
\hline
\end{tabular}
\end{table}

The 98.43\% alignment underscores dataset complementarity, despite minor discrepancies from collection timing, late-season additions, or minimal-involvement filtering. An inner join on matches proceeded accordingly.

\paragraph{Step 5: Final Integration}

Filtered datasets merged on (name, gameweek):

\begin{lstlisting}[language=Python, caption={Final Dataset Integration}]
# Filter to matching pairs
matching_pairs = olb_pairs & vaa_pairs

olb_filtered = olb_df[
    olb_df.apply(lambda row: (row['player_name'], row['gw']) 
                 in matching_pairs, axis=1)]

vaa_filtered = vaa_df[
    vaa_df.apply(lambda row: (row['name'], row['gw']) 
                 in matching_pairs, axis=1)]

# Merge on (name, gw)
integrated_df = pd.merge(olb_filtered, vaa_filtered,
                        left_on=['player_name', 'gw'],
                        right_on=['name', 'gw'],
                        how='inner')
\end{lstlisting}

The resulting \texttt{integrated\_df.csv} holds 27,222 records across 16 features. From OLB: \texttt{status}, \texttt{chance\_of\_playing\_*}, \texttt{now\_cost}, \texttt{event\_points}, \texttt{ep\_*}. From VAA: \texttt{position}, \texttt{team}, \texttt{xP}, \texttt{minutes}, \texttt{starts}, \texttt{total\_points}, \texttt{value}. Shared keys: \texttt{name}, \texttt{gw}.


\subsubsection{Integration Results and Data Quality}

The integration yielded a unified, high-quality dataset spanning the 2024-25 season (GW 1--38), with 27,222 unique player-gameweek records and 98.43\% completeness across matching pairs from independent sources. It encompasses 16 features addressing pricing, performance, availability, and predictions, free of duplicates.

Leveraging complementary strengths, the dataset fuses OLB's pricing details with VAA's performance metrics. This synthesis empowers the optimization model to guide both initial squad assembly and ongoing transfer strategies.

The integration phase established a robust foundation for subsequent data cleaning phases, with documented data provenance and validated consistency across sources.


\subsection{Phase 2: Data Cleaning}

Although data integration merged complementary sources effectively, the resulting \texttt{integrated\_df.csv} (27,222 records, 16 features) harbored conflicting measurements from independent origins. Cleaning addressed three objectives: reconciling discrepancies, ensuring consistency, and forging unified features for optimization.

\subsubsection{Cost Reconciliation}

Pricing data appeared dually: OLB's \texttt{now\_cost} (scaled ×10) and VAA's \texttt{value}, both denoting £0.1M buying prices, potentially captured at disparate intra-gameweek moments.

Of 27,222 records, 1,522 (5.59\%) diverged, with differences spanning -2.0 to +2.0 units (±£0.2M; mean 0.022, SD 0.240). Discrepancies peaked in GW2 (70 cases, 11.16\%) and GW13 (69 cases, 10.00\%).

A conservative strategy adopted the maximum:

\begin{lstlisting}[language=Python, caption={Cost Reconciliation Strategy}]
# Create reconciled cost using maximum value
df['cost_reconciled'] = np.maximum(df['now_cost'] * 10, df['value'])
\end{lstlisting}

This yields cautious estimates, averting selections premised on understated prices, and unifies pricing.

\subsubsection{Points Validation}

Gameweek points manifested as OLB's \texttt{event\_points} and VAA's \texttt{total\_points}, proxying identical FPL earnings.

Exact alignment prevailed across all 27,222 records, validating source fidelity to official scores and integration precision.

\subsubsection{Expected Points Reconciliation}

Expected points varied: OLB's \texttt{ep\_this} versus VAA's \texttt{xP}.

Of records, 23,515 (86.38\%) matched precisely; 3,707 (13.62\%) diverged (-22.0 to +26.4; mean -0.061, SD 0.907). Zeros in \texttt{xP} afflicted GW22, 32, and 34 entirely, driving peaks: GW32 (520, 65.16\%), GW22 (464, 63.65\%), GW34 (394, 61.09\%). Mismatches occurred in 16 of 38 gameweeks.

Leveraging actual \texttt{total\_points}, reconciliation favored the lower absolute error:

\begin{lstlisting}[language=Python, caption={Expected Points Reconciliation Logic}]
# Calculate absolute differences from actual performance
diff_ep_this = np.abs(df['ep_this'] - df['total_points'])
diff_xP = np.abs(df['xP'] - df['total_points'])

# Select prediction with smaller error
df['ep_reconciled'] = np.where(diff_ep_this <= diff_xP, 
                                df['ep_this'], 
                                df['xP'])
\end{lstlisting}


Note: This retrospective method suits backtesting but not prospective use, where outcomes are unknown. 

\subsubsection{Player Set Standardization}

Participation varied: 417 players spanned all 38 gameweeks; 116 in 37; 105 in 36; others, fewer. Retention targeted $\geq$36 gameweeks, yielding ~200 supplementary players for consistency.

Absences for the 221 near-complete players (36-37 gws) clustered non-consecutively in GW1, 15, 29, 34:

\begin{table}[h]
\centering
\caption{Distribution of Missing Gameweeks for Players with Near-Complete Participation}
\begin{tabular}{|c|c|c|c|c|}
\hline
\textbf{GW 1} & \textbf{GW 15} & \textbf{GW 29} & \textbf{GW 34} & \\
\hline
7 & 26 & 33 & 50 & \\
\hline
\textbf{GW(1,2)} & \textbf{GW(1,15)} & \textbf{GW(1,34)} & \textbf{GW(15,29)} & \textbf{GW(29,34)} \\
\hline
11 & 2 & 2 & 28 & 62 \\
\hline
\end{tabular}
\end{table}

Imputation tailored by variable: Costs interpolated linearly (preceding/following averages, rounded; edges via nearest available) for GW3--38 continuity. Performance metrics (\texttt{points}, \texttt{eP}, \texttt{prob\_showup}, \texttt{minutes}) zeroed; an \texttt{unavailable} flag distinguished imputations.

This culled 166 sporadic players, standardizing to 638 across 38 gameweeks (326 imputations), easing modeling.


\subsubsection{Cleaning Results and Data Quality}

Cleaning refined \texttt{integrated\_df.csv} into \texttt{cleaned\_data.csv}: 24,244 records (ouf of 27222, `   89.0\% retention), 638 players/gameweek, fully consistent.

Features: \texttt{player\_id}, \texttt{gw}, \texttt{name}, \texttt{position}, \texttt{team}, \texttt{cost} (reconciled), \texttt{points}, \texttt{eP} (reconciled), \texttt{minutes}, \texttt{unavailable}.

The final dataset provides a clean, consistent foundation for subsequent modeling phases, with documented reconciliation strategies, validated data quality, and engineered features designed specifically for optimization applications.

\subsection{Phase 3: MILP Model Implementation}

The Mixed-Integer Linear Programming (MILP) implementation converts the mathematical formulation from Section 3 into executable code via the PuLP library in Python. PuLP was chosen for its declarative constraint syntax, compatibility with diverse solvers, and integration with Python's data tools. It adheres to the two-phase model: Pre-GW1 optimization for initial squad assembly and Post-GW1 for weekly transfers and lineups.

Modularity structures the code into core elements: optimization modules (\texttt{pre\_gw1\_step1.py}, \texttt{pre\_gw1\_step2.py}, 

\texttt{post\_gw1\_step1.py}) for model execution; a utilities module (\texttt{utils.py}) for post-hoc tasks; and a season orchestrator (\texttt{run\_season\_optimizer.py}) overseeing the 38-gameweek simulation. This separation isolates model logic from workflow management, facilitating targeted testing and validation.


\subsubsection{Data Structures and Flow}

The MILP implementation employs a structured data flow architecture to coordinate optimization decisions while preserving persistent state across gameweeks.
\newpage
\paragraph{Input Data Structure}

All optimization functions ingest player data through a pandas DataFrame conforming to the specified schema:

\begin{table}[h]
\centering
\caption{Player DataFrame Schema for MILP Optimization}
\begin{tabular}{|l|l|p{6cm}|}
\hline
\textbf{Column} & \textbf{Type} & \textbf{Description} \\
\hline
\texttt{player\_id} & int & Unique identifier (1-638) \\
\texttt{gw} & int & Gameweek number (1-38) \\
\texttt{name} & str & Player name \\
\texttt{position} & str & Position (GK, DEF, MID, FWD) \\
\texttt{team} & str & Club name \\
\texttt{cost} & int & Price in tenths of millions (55 = £5.5m) \\
\texttt{eP} & float & Expected points prediction \\
\texttt{points} & int & Actual points scored (for oracle mode) \\
\texttt{unavailable} & int & Availability flag (0=available, 1=unavailable) \\
\hline
\end{tabular}
\end{table}

\paragraph{State Variables}

The Post-GW1 optimization sustains four state variables to propagate essential information across gameweeks:

\begin{table}[h]
\centering
\caption{State Variables for Post-GW1 Optimization}
\begin{tabular}{|l|l|p{7cm}|}
\hline
\textbf{Variable} & \textbf{Type} & \textbf{Description} \\
\hline
\texttt{y0} & dict[int $\rightarrow$ \{0,1\}] & Current squad (638 binary values) \\
\texttt{p0} & dict[int $\rightarrow$ int] & Purchase prices in tenths (e.g., 55 = £5.5m) \\
\texttt{B\_bank} & int & Cash in bank in tenths (e.g., 50 = £5.0m) \\
\texttt{f} & int & Free transfers available (0-5) \\
\hline
\end{tabular}
\end{table}

These state variables are initialized following GW1 optimization and updated thereafter after each gameweek in accordance with the transfer decisions.

\paragraph{Decision Variable Output}

Each optimization yields decision variables as dictionaries mapping player IDs to binary values:

\begin{table}[h]
\centering
\caption{Decision Variable Outputs}
\begin{tabular}{|l|l|p{7cm}|}
\hline
\textbf{Variable} & \textbf{Type} & \textbf{Description} \\
\hline
\texttt{x} & dict[int $\rightarrow$ \{0,1\}] & Starting XI (11 players = 1) \\
\texttt{y} & dict[int $\rightarrow$ \{0,1\}] & Squad (15 players = 1) \\
\texttt{c} & dict[int $\rightarrow$ \{0,1\}] & Captain (1 player = 1) \\
\texttt{v} & dict[int $\rightarrow$ \{0,1\}] & Vice-captain (1 player = 1, post-hoc) \\
\texttt{b1-b4} & dict[int $\rightarrow$ \{0,1\}] & Bench positions (1 per position, post-hoc) \\
\hline
\end{tabular}
\end{table}

All decision variables encompass the full 638-player set, assigning 0 or 1 to each player ID to enable uniform JSON serialization.

\subsubsection{Pre-GW1 Implementation}

The Pre-GW1 optimization employs a hierarchical approach.

\paragraph{Step 1: Starter and Captain Selection}

The initial phase (\texttt{pre\_gw1\_step1.py}) optimizes the starting XI and captaincy under budget and squad constraints. Its objective maximizes expected points from starters, augmented by the captain's double bonus:

\begin{lstlisting}[language=Python, caption={Pre-GW1 Step 1 Objective Function}]
# Decision variables
y = pulp.LpVariable.dicts("squad", player_ids, cat='Binary')
x = pulp.LpVariable.dicts("starter", player_ids, cat='Binary')
c = pulp.LpVariable.dicts("captain", player_ids, cat='Binary')
# Objective: Maximize expected points from starters and captain
prob += pulp.lpSum([
    players[pid]['expected_points'] * (x[pid] + c[pid])
    for pid in player_ids
]), "Total_Expected_Points"
\end{lstlisting}

The constraint set comprehensively enforces all FPL rules governing team selection:

\begin{lstlisting}[language=Python, caption={Pre-GW1 Step 1: Comprehensive Constraint Implementation}]
# Lineup constraints: 11 starters with position requirements
prob += pulp.lpSum([x[pid] for pid in player_ids]) == 11, "Lineup_Size"
prob += pulp.lpSum([x[pid] for pid in gk_ids]) == 1, "Starting_GK"
prob += pulp.lpSum([x[pid] for pid in def_ids]) >= 3, "Min_Starting_DEF"
prob += pulp.lpSum([x[pid] for pid in def_ids]) <= 5, "Max_Starting_DEF"
prob += pulp.lpSum([x[pid] for pid in mid_ids]) >= 2, "Min_Starting_MID"
prob += pulp.lpSum([x[pid] for pid in mid_ids]) <= 5, "Max_Starting_MID"
prob += pulp.lpSum([x[pid] for pid in fwd_ids]) >= 1, "Min_Starting_FWD"
prob += pulp.lpSum([x[pid] for pid in fwd_ids]) <= 3, "Max_Starting_FWD"
# Captain constraints: exactly 1, must be starter
prob += pulp.lpSum([c[pid] for pid in player_ids]) == 1, "One_Captain"
for pid in player_ids:
    prob += c[pid] <= x[pid], f"Captain_Must_Start_{pid}"
# Unavailable players cannot start
for pid in unavailable_ids:
    prob += x[pid] == 0, f"Unavailable_Cannot_Start_{pid}"
# Squad constraints: 15 players with position quotas
prob += pulp.lpSum([y[pid] for pid in player_ids]) == 15, "Squad_Size"
prob += pulp.lpSum([y[pid] for pid in gk_ids]) == 2, "Squad_GK"
prob += pulp.lpSum([y[pid] for pid in def_ids]) == 5, "Squad_DEF"
prob += pulp.lpSum([y[pid] for pid in mid_ids]) == 5, "Squad_MID"
prob += pulp.lpSum([y[pid] for pid in fwd_ids]) == 3, "Squad_FWD"
# Budget constraint: total cost <= 100M (1000 in tenths)
prob += pulp.lpSum([
    players[pid]['cost'] * y[pid] for pid in player_ids
]) <= 1000, "Budget_Limit"
# Club limits: max 3 players per club
for club in clubs:
    club_players = [pid for pid in player_ids if players[pid]['team'] == club]
    prob += pulp.lpSum([y[pid] for pid in club_players]) <= 3, f"Club_Limit_{club}"
# Starters must be in squad
for pid in player_ids:
    prob += x[pid] <= y[pid], f"Starter_In_Squad_{pid}"
\end{lstlisting}

Upon solving, Pre-GW1 Step 1 yields the optimal starting XI and captain:

\begin{lstlisting}[language=Python, caption={Pre-GW1 Step 1 Return Values}]
solution = {
    'x': {pid: int(round(x[pid].varValue)) for pid in player_ids},  # Starters
    'c': {pid: int(round(c[pid].varValue)) for pid in player_ids}   # Captain 
    }
\end{lstlisting}

\paragraph{Step 2: Bench Optimization}

The subsequent phase (\texttt{pre\_gw1\_step2.py}) optimizes bench composition, fixing the starting XI from Step 1 via equality constraints. This simplifies the constraints to budget, squad quotas, and club limits.

\begin{lstlisting}[language=Python, caption={Pre-GW1 Step 2: Reduced Constraint Set}]
# Fix starters from Step 1
for pid in fixed_starter_ids:
    prob += y[pid] == 1, f"Fix_Starter_{pid}"
# Objective: Maximize total expected points of squad
prob += pulp.lpSum([
    players[pid]['expected_points'] * y[pid]
    for pid in player_ids
]), "Total_Squad_Expected_Points"
# Reduced constraints are omitted for simplicity
\end{lstlisting}

This method optimizes bench composition to enhance squad quality while preserving the optimal starting XI.

Prior to returning the solution, state variables are initialized for GW2:

\begin{lstlisting}[language=Python, caption={State Initialization After GW1}]
# Purchase prices: record initial buying prices
p0_buy = {
    pid: (players[pid]['cost'] if y_star[pid] == 1 else 0)
    for pid in player_ids
}
# Bank balance: remaining budget after squad purchase
total_cost = sum(players[pid]['cost'] * y_star[pid] for pid in player_ids)
B_bank = 1000 - total_cost  # 1000 = 100M in tenths
# Free transfers: 2 for GW2 (1 per week + 1 banked)
f = 2
\end{lstlisting}

The function subsequently returns the optimized squad alongside the initialized state variables:

\begin{lstlisting}[language=Python, caption={Pre-GW1 Step 2 Return Values}]
solution = {
    # Final squad
    'y': {pid: int(round(y[pid].varValue)) for pid in player_ids},  
    # Initial state for GW2
    'y0': y_star,  # Squad becomes initial state
    'p0': p0_buy,  # Purchase prices
    'B_bank': B_bank,  # Cash in bank
    'f': 2  # Free transfers for GW2
}
\end{lstlisting}

\subsubsection{Post-GW1 Implementation}

The Post-GW1 optimization (\texttt{post\_gw1\_step1.py}) implements the transfer-enabled formulation.

\paragraph{Selling Price Calculation}

FPL's 50\% profit lock governs selling prices: upon appreciation, only half the profit is recouped, ceiling-rounded to the nearest £0.1M:

\begin{lstlisting}[language=Python, caption={FPL 50\% Profit Lock Implementation}]
selling_prices = {}
for pid in player_ids:
    current_price = players[pid]['cost']
    purchase_price = p0[pid]
    if current_price > purchase_price:
        profit = current_price - purchase_price
        # Lock 50% profit, round up (FPL always rounds up)
        selling_prices[pid] = purchase_price + math.ceil(0.5 * profit)
    else:
        # Loss: recover full current price
        selling_prices[pid] = current_price
\end{lstlisting}

This mechanism ensures that players sold at a gain yield the purchase price plus half the profit, whereas those sold at a loss recover their full current market price.
\paragraph{Dynamic Budget Constraint}

The budget constraint enforces non-negative cash balances following transfers. Only purchases of new players incur costs, as retained players are already owned:

\begin{lstlisting}[language=Python, caption={Post-GW1 Dynamic Budget Constraint}]
# Decision variables for transfers
t = pulp.LpVariable.dicts("transfer_in", player_ids, cat='Binary')
s = pulp.LpVariable.dicts("transfer_out", player_ids, cat='Binary')
# Cost of new purchases <= cash in bank + selling proceeds
prob += pulp.lpSum([
    players[pid]['cost'] * t[pid]
    for pid in player_ids
]) <= B_bank + pulp.lpSum([
    selling_prices[pid] * s[pid]
    for pid in player_ids
]), "Budget_Limit"
\end{lstlisting}

This constraint dynamically updates the available budget in response to transfer decisions, permitting flexible trading within fiscal limits.

\paragraph{Transfer Logic Constraints}

Transfer mechanics are enforced via an interconnected suite of constraints:

\begin{lstlisting}[language=Python, caption={Transfer Logic Constraints}]
# Squad update: new squad = old squad + buys - sells
for pid in player_ids:
    prob += y[pid] == y0[pid] + t[pid] - s[pid], f"Squad_Update_{pid}"
# No simultaneous buy and sell of same player
for pid in player_ids:
    prob += t[pid] + s[pid] <= 1, f"No_Simultaneous_Transfer_{pid}"
# Can only buy players not in current squad
for pid in player_ids:
    prob += t[pid] <= 1 - y0[pid], f"Buy_Only_Non_Squad_{pid}"
# Can only sell players in current squad  
for pid in player_ids:
    prob += s[pid] <= y0[pid], f"Sell_Only_Current_Squad_{pid}"
\end{lstlisting}

\paragraph{Transfer Penalty Calculation}

The penalty for excess transfers is calculated at four points per transfer beyond the free allowance:

\begin{lstlisting}[language=Python, caption={Extra Transfer Penalty}]
# e: number of extra transfers beyond free_transfers
e = pulp.LpVariable("extra_transfers", lowBound=0, upBound=15, cat='Integer')
# Extra transfers = max(0, total_transfers - free_transfers)
prob += e >= pulp.lpSum([t[pid] for pid in player_ids]) - f, "Extra_Transfers"
# Objective: Expected points minus 4-point penalty per extra transfer
prob += pulp.lpSum([
    players[pid]['expected_points'] * (x[pid] + c[pid])
    for pid in player_ids
]) - 4 * e, "Total_Expected_Points_Minus_Penalty"
\end{lstlisting}

\paragraph{State Update}

Following optimization, state variables are updated for the subsequent gameweek:

\begin{lstlisting}[language=Python, caption={State Update After Post-GW1 Optimization}]
# Update purchase prices
for pid in player_ids:
    if s_star[pid]:  # Sold player
        p0[pid] = 0
    elif t_star[pid]:  # Bought player
        p0[pid] = players[pid]['cost']
    # Otherwise unchanged
# Update bank balance
total_selling_proceeds = sum(selling_prices[pid] * s_star[pid] 
                             for pid in player_ids)
total_purchase_costs = sum(players[pid]['cost'] * t_star[pid] 
                           for pid in player_ids)
B_bank = B_bank + total_selling_proceeds - total_purchase_costs
# Update free transfers (remaining from previous + 1 new per week, max 5)
transfers_used = sum(t_star.values())
remaining_free = max(0, f - transfers_used)
f_new = min(5, remaining_free + 1)
\end{lstlisting}

\subsubsection{Post-Hoc Processing}

Vice-captain and bench position assignments follow post-optimization greedy algorithm in \texttt{utils.py}.

\paragraph{Vice-Captain Selection}

The vice-captain is designated as the non-captain starter with the highest expected points:

\begin{lstlisting}[language=Python, caption={Vice-Captain Selection}]
vice_captain_id = max(
    (pid for pid in player_ids if x_star[pid] == 1 and c_star[pid] == 0),
    key=lambda pid: players[pid]['expected_points']
)
\end{lstlisting}

\paragraph{Bench Position Assignment}

Bench ordering adheres to FPL substitution priorities:

\begin{lstlisting}[language=Python, caption={Bench Position Assignment}]
# Position 1: Backup goalkeeper
bench_gks = [pid for pid in bench_ids if players[pid]['position'] == 'GK']
gk_id = bench_gks[0]
# Positions 2-4: Outfield sorted by expected points (descending)
outfield_ids = [pid for pid in bench_ids if players[pid]['position'] != 'GK']
outfield_sorted = sorted(outfield_ids, 
                         key=lambda pid: players[pid]['expected_points'], 
                         reverse=True)
# Bench order: [pos1_GK, pos2_best, pos3_mid, pos4_worst]
bench_order = [gk_id] + outfield_sorted
\end{lstlisting}

\subsubsection{Season Orchestration}

The season optimizer (\texttt{run\_season\_optimizer.py}) orchestrates optimization across all 38 gameweeks.

\paragraph{Oracle Mode}

In oracle mode, conditional expected points are substituted with actual points scored for each player. This facilitates perfect-information backtesting, yielding performance upper bounds.

\begin{lstlisting}[language=Python, caption={Oracle Mode Implementation}]
players_df = players_df.copy()
if oracle:
    # Oracle mode: replace expected points with actual points
    players_df['expected_points'] = players_df['points']
else:
    # Normal mode: use predicted expected points
    players_df['expected_points'] = players_df['eP']
\end{lstlisting}

\paragraph{Workflow Management}

The orchestrator oversees the complete workflow:

\begin{lstlisting}[language=Python, caption={Season Optimization Workflow}]
for gw in range(1, 39):
    players_gw = all_data[all_data['gw'] == gw].copy()
    if gw == 1:
        # Pre-GW1: Three-step pipeline
        step1 = optimize_pre_gw1_step1(players_gw, solver=solver)
        step2 = optimize_pre_gw1_step2(players_gw, step1['x'], solver=solver)
        post_hoc = process_pre_gw1_solution(players_gw, step1['x'], 
                                            step2['y'], step1['c'])
        # Initialize state variables
        y0, p0, B_bank, f = step2['y0'], step2['p0'], step2['B_bank'], step2['f']
    else:
        # Post-GW1: Two-step pipeline
        step1 = optimize_post_gw1_step1(players_gw, y0, p0, f, B_bank, 
                                        solver=solver)
        post_hoc = process_post_gw1_solution(players_gw, step1['x'], 
                                             step1['y'], step1['c'])
        # Update state variables
        y0, p0, B_bank, f = step1['y0'], step1['p0'], step1['B_bank'], step1['f']
\end{lstlisting}

\subsubsection{Technical Challenges and Solutions}

\paragraph{Floating-Point Precision}

PuLP solvers return binary variables as floating-point values (e.g., 0.9999999 instead of 1.0). This caused errors in downstream processing expecting exact integers. Rounding to the nearest integer resolves this:

\begin{lstlisting}[language=Python, caption={Binary Variable Precision Handling}]
# Extract optimal solution and force to exact integers
x_star = {pid: int(round(x[pid].varValue)) for pid in player_ids}
y_star = {pid: int(round(y[pid].varValue)) for pid in player_ids}
\end{lstlisting}

\paragraph{Solver Selection and Compatibility}

A unified interface supports multiple solvers:

\begin{lstlisting}[language=Python, caption={Solver Selection Interface}]
def get_pulp_solver(solver_name='CBC', msg=0):
    solvers = {
        'CBC': pulp.PULP_CBC_CMD(msg=msg),
        'GLPK': pulp.GLPK_CMD(msg=msg),
        'SCIP': pulp.SCIP_PY(msg=msg)
    }
    return solvers.get(solver_name.upper(), pulp.PULP_CBC_CMD(msg=msg))
\end{lstlisting}

\subsubsection{Implementation Results}

The MILP implementation sequentially optimizes all 38 gameweeks with persistent 
state propagation. Its modular architecture facilitated isolated validation, 
achieving full test coverage for Pre-GW1 and Post-GW1 pipelines. Optimization 
across three solvers generated consistent results.

Output comprises standardized JSON files with 12 keys per gameweek: eight binary 
decision dictionaries (\texttt{x}, \texttt{y}, \texttt{c}, \texttt{v}, 
\texttt{b1}--\texttt{b4}) mapping 638 player IDs to \{0,1\}, plus four state 
variables (\texttt{y0}, \texttt{p0}, \texttt{B\_bank}, \texttt{f}), promoting 
data consistency and integration with backtesting tools.

\subsection{Phase 4: CP MiniZinc Implementation}

TThe Constraint Programming formulation, realized in MiniZinc, provides an alternative 
paradigm for evaluating different solvers. 
MiniZinc's declarative syntax enables set-based representations and global constraints, 
while CP solvers emphasize constraint propagation over MILP's linear relaxations. 
The implementation mirrors MILP's modular architecture with three optimization 
modules for Pre-GW1 Steps 1-2 and Post-GW1. Each integrates a Python wrapper for 
data conversion and solver interfacing with a MiniZinc model for declarative 
constraint formulation.

\subsubsection{Pre-GW1: Set-Based Formulation}

The initial step optimizes the starting XI and captain selection, paralleling the MILP Pre-GW1 Step 1 but employing set-based decision variables in place of binary arrays.

\paragraph{Set Variables}

In the Pre-GW1 Step 1 model, the core decision variables are defined as sets rather than indexed binary variables:

\begin{lstlisting}[language=Python, caption={MiniZinc Set-Based Decision Variables (pre\_gw1\_step1.mzn)}]
% Decision Variables: Sets and Set-of-Sets (Partitions)
var set of Players: Squad;          % Q: Squad set, |Q|=15
var set of Players: Starters;       % S subset Q: Starting XI, |S|=11
var set of Players: Captain;        % Singleton set for captain

array[Position] of var set of Players: SquadPartitions;
array[Position] of var set of Players: StarterPartitions;

% Subset Constraints 
constraint Starters subset Squad;
constraint Captain subset Starters;
\end{lstlisting}

This formulation encapsulates the problem structure directly: the squad comprises a set of players, the starters a subset of the squad, and the captain a singleton subset of the starters. Set membership is articulated through subset relationships.

\paragraph{Partitioning into Disjoint Sets by Position}

To enforce FPL positional requirements, the model partitions the squad and starting XI into disjoint position-specific subsets. A partition consists of mutually exclusive subsets that collectively encompass the entire set, assigning each player to exactly one position category.

\begin{lstlisting}[language=Python, caption={Disjoint Partition Constraints (pre\_gw1\_step1.mzn)}]
% Partition Constraints (disjoint union, subsets of available players)
constraint forall(p1, p2 in Position where p1 < p2) (disjoint(SquadPartitions[p1], SquadPartitions[p2]));
constraint Squad = array_union([SquadPartitions[p] | p in Position]);
constraint forall(p in Position) (SquadPartitions[p] subset PosPlayers[p]);

constraint forall(p1, p2 in Position where p1 < p2) (disjoint(StarterPartitions[p1], StarterPartitions[p2]));
constraint Starters = array_union([StarterPartitions[p] | p in Position]);
constraint forall(p in Position) (StarterPartitions[p] subset SquadPartitions[p]);
\end{lstlisting}

\paragraph{Cardinality} FPL squad and starting XI composition requirements across four positions are enforced through cardinality constraints on each partition. These are implemented via the \texttt{card()} function, which computes the number of elements in a set:

\begin{lstlisting}[language=Python, caption={Cardinality Constraints for FPL Rules (pre\_gw1\_step1.mzn)}]
% Cardinalities
constraint card(Squad) = 15;
constraint card(Starters) = 11;
constraint card(Captain) = 1;
% Position-Specific Cardinalities
constraint card(SquadPartitions[GK]) = 2 /\ card(StarterPartitions[GK]) = 1;
constraint card(SquadPartitions[DEF]) = 5 /\ 3 <= card(StarterPartitions[DEF]) /\ card(StarterPartitions[DEF]) <= 5;
constraint card(SquadPartitions[MID]) = 5 /\ 2 <= card(StarterPartitions[MID]) /\ card(StarterPartitions[MID]) <= 5;
constraint card(SquadPartitions[FWD]) = 3 /\ 1 <= card(StarterPartitions[FWD]) /\ card(StarterPartitions[FWD]) <= 3;
\end{lstlisting}

\paragraph{Integer Arithmetic and Scaling}

The MiniZinc implementation mandates integer arithmetic exclusively. Expected points are scaled by a factor of 10 to eliminate fractions.


\paragraph{Step 2: Bench Optimization}

The subsequent step optimizes bench composition, with the starting XI from Step 1 held fixed, paralleling the MILP Pre-GW1 Step 2 formulation.

\begin{lstlisting}[language=Python, caption={Pre-GW1 Step 2: Fixed Starters Constraint (pre\_gw1\_step2.mzn)}]
% Fix Starters and Partitions from Step 1
constraint Starters = FixedStarters;
constraint forall(p in Position) (StarterPartitions[p] = FixedStarterPartitions[p]);
\end{lstlisting}

This optimization designates the Step 1 starters as mandatory squad members, then selects 4 additional players under the identical constraints, with the 11 starters pre-fixed.

\subsubsection{Post-GW1: Transfer-Enabled Optimization}

The Post-GW1 model incorporates added complexity via transfer mechanics. An initial set-difference method (deriving buys as \texttt{Squad diff OldSquad}) was explored but abandoned due to unresolved implementation challenges; the final formulation employs explicit binary variables for transfers to guarantee reliable constraint propagation:

\begin{lstlisting}[language=Python, caption={Explicit Transfer Variables (post\_gw1\_step1.mzn)}]
% Explicit Transfer Variables
array[Players] of var bool: t;  % t[i] = 1 if buy i (transfer in)
array[Players] of var bool: s;  % s[i] = 1 if sell i (transfer out)
% Squad update: y_i = y0_i + t_i - s_i
constraint forall(i in Players) (
    y[i] = bool2int(y0[i]) + bool2int(t[i]) - bool2int(s[i])
);
\end{lstlisting}

\paragraph{Set-Array Channeling}

A key technical challenge in the Post-GW1 model involves aligning set-based squad representations with array-based transfer variables. MiniZinc's channeling constraints establish bidirectional consistency between these paradigms:

\begin{lstlisting}[language=Python, caption={Set-Array Channeling (post\_gw1\_step1.mzn)}]
% Decision variables in both representations
array[Players] of var 0..1: y;  % Squad membership as array
var set of Players: Squad;       % Squad as set
% Channeling: link int array y with set variable Squad
constraint forall(i in Players) (
    (y[i] = 1) <-> (i in Squad)
);
\end{lstlisting}

The equivalence constraint \texttt{(y[i] = 1) <-> (i in Squad)} links binary array element \( y[i] \) to squad membership: player \( i \) resides in the Squad set if and only if \( y[i] = 1 \). This channeling permits simultaneous application of set-based constraints—for subset relations and global rules—with array-based constraints for transfer computations.

\paragraph{Transfer Constraints}

Transfer logic constraints parallel the MILP formulation, adapted to MiniZinc's boolean type system:

\begin{lstlisting}[language=Python, caption={Transfer Logic (post\_gw1\_step1.mzn)}]
% Cannot buy and sell the same player
constraint forall(i in Players) (
    not (t[i] /\ s[i])
);
% Can only buy players not in current squad
constraint forall(i in Players) (
    y0[i] -> not t[i]
);
% Can only sell players in current squad
constraint forall(i in Players) (
    not y0[i] -> not s[i]
);
\end{lstlisting}

The logical implication operator \texttt{->} enables more concise constraint formulations than the inequality-based encodings required in MILP.

\subsubsection{Python-MiniZinc Integration}

The implementation leverages the \texttt{minizinc} Python package to integrate Python's data processing ecosystem with MiniZinc's constraint solver.

\paragraph{Model Loading and Solver Selection}

Each optimization module loads the corresponding MiniZinc model and instantiates a solver:

\begin{lstlisting}[language=Python, caption={MiniZinc Model and Solver Initialization}]
from minizinc import Instance, Model, Solver
# Load the MiniZinc model
model_path = os.path.join(os.path.dirname(__file__), 'pre_gw1_step1.mzn')
model = Model(model_path)
# Get solver (supports cp-sat, gecode, chuffed)
try:
    solver = Solver.lookup('cp-sat')  # OR-Tools CP-SAT (default)
except LookupError:
    solver = Solver.lookup('gecode')  # Fallback to Gecode
# Create instance
instance = Instance(solver, model)
\end{lstlisting}

OR-Tools CP-SAT was used as the default solver, with Gecode and Chuffed also tested for validation.

\subsubsection{Technical Challenges and Solutions}

\paragraph{Parameter Conversion and Array Indexing}

A key technical nuance is MiniZinc's 1-based array indexing, contrasting Python's 0-based convention. The data preparation module manages this conversion:

\begin{lstlisting}[language=Python, caption={1-Based Array Construction (data\_prep.py)}]
# Build arrays with sequential 1-based indices
expected_points = [0] * (n_players + 1)  # Index 0 unused
cost = [0] * (n_players + 1)
unavailable = [False] * (n_players + 1)
for pid in player_ids:
    idx = player_id_to_idx[pid]  # 1-based index
    row = players_df[players_df['player_id'] == pid].iloc[0]
    expected_points[idx] = int(round(float(row['expected_points']) * 10))
    cost[idx] = int(round(float(row['cost'])))
    unavailable[idx] = bool(row['unavailable'])
# Return arrays excluding index 0 (MiniZinc expects 1..n_players)
params = {
    'expected_points': expected_points[1:],  # 1..n_players
    'cost': cost[1:],
    'unavailable': unavailable[1:]
}
\end{lstlisting}

\paragraph{Result Parsing and Set Conversion}

Post-solving, MiniZinc yields set variables as Python sets of 1-based indices, which require conversion to binary dictionaries mapping player IDs to 0/1 values:

\begin{lstlisting}[language=Python, caption={Set to Binary Dictionary Conversion (data\_prep.py)}]
# Get sets from MiniZinc output
squad_set = result['Squad']         # Set of 1-based indices
starters_set = result['Starters']
captain_set = result['Captain']
# Create mapping from 1-indexed position to player_id
player_ids = players_df['player_id'].tolist()
# Convert sets to binary dictionaries
y = {pid: (1 if (idx + 1) in squad_set else 0) 
     for idx, pid in enumerate(player_ids)}
x = {pid: (1 if (idx + 1) in starters_set else 0) 
     for idx, pid in enumerate(player_ids)}
c = {pid: (1 if (idx + 1) in captain_set else 0) 
     for idx, pid in enumerate(player_ids)}
\end{lstlisting}

This conversion aligns the CP output with the MILP format, facilitating uniform post-hoc processing and backtesting pipelines.

\paragraph{Upper Bound Calculation}

CP solvers leverage tight upper bounds on objective variables to accelerate search processes. The data preparation module derives problem-specific bounds accordingly.
\begin{lstlisting}[language=Python, caption={Upper Bound Calculation (data\_prep.py)}]
def calculate_upper_bound(players_df):
    # Get available players only
    available_df = players_df[players_df['unavailable'] == 0].copy()
    
    # Best 11 starters by expected points
    best_11 = available_df.nlargest(11, 'expected_points')['expected_points'].sum()
    captain_bonus = available_df['expected_points'].max()
    
    ub = best_11 + captain_bonus
    # Scale by 10 for integer arithmetic, add buffer
    return int(ub * 10) + 10
\end{lstlisting}

This optimistic estimate assumes selecting the 11 highest-expected-points players, disregarding positional and squad constraints. Though unattainable, it furnishes a reliable upper limit for effective pruning.

\subsubsection{Implementation Results}

The MiniZinc CP implementation satisfied all validation criteria: comprehensive optimization across 38 gameweeks with persistent state, and output aligned with the MILP format to facilitate direct comparison.
Execution encompassed three solvers (OR-Tools CP-SAT—default—Gecode, Chuffed), affirming independence across backends.
\newpage

\subsection{Phase 5: Backtesting and Validation}

The backtesting phase evaluates optimizer performance against actual gameweek 
outcomes, incorporating FPL rule validation as sanity checks. The backtester 
ingests season-long JSON outputs (38 gameweek decisions) and emulates gameplay 
via automatic substitutions, captain doubling, and transfer tracking, yielding 
metrics including total points, per-gameweek scores, and transfer efficiency 
alongside compliance verification. Implementation comprises three modules: 
\texttt{fpl\_validator.py} for rule checks, \texttt{fpl\_point\_calculator.py} for substitutions 
and scoring, and \texttt{backtester.py} for season orchestration, with modularity 
supporting isolated testing and cross-optimizer reuse.
\subsubsection{Architecture and Data Flow}

The backtester operates on two core data sources: optimizer output JSON files and the cleaned dataset.

\paragraph{Input: Optimizer Output Format}

Each optimizer generates a standardized JSON file encapsulating decisions for all 38 gameweeks. Per gameweek, it features 12 keys:

\begin{table}[h]
\centering
\caption{Optimizer Output Structure Per Gameweek}
\begin{tabular}{|l|l|p{7cm}|}
\hline
\textbf{Key} & \textbf{Type} & \textbf{Description} \\
\hline
\texttt{x} & dict[player\_id $\rightarrow$ \{0,1\}] & Starting XI (11 players = 1, rest = 0) \\
\texttt{y} & dict[player\_id $\rightarrow$ \{0,1\}] & Squad (15 players = 1, rest = 0) \\
\texttt{c} & dict[player\_id $\rightarrow$ \{0,1\}] & Captain (1 player = 1, rest = 0) \\
\texttt{v} & dict[player\_id $\rightarrow$ \{0,1\}] & Vice-captain (1 player = 1, rest = 0) \\
\texttt{b1-b4} & dict[player\_id $\rightarrow$ \{0,1\}] & Bench positions (1 player per position) \\
\texttt{y0} & dict[player\_id $\rightarrow$ \{0,1\}] & Previous squad state \\
\texttt{p0} & dict[player\_id $\rightarrow$ int] & Purchase prices (tenths of millions) \\
\texttt{B\_bank} & int & Cash in bank (tenths of millions) \\
\texttt{f} & int & Free transfers available (0-5) \\
\hline
\end{tabular}
\end{table}

All binary decision variables (\texttt{x}, \texttt{y}, \texttt{c}, \texttt{v}, \texttt{b1}--\texttt{b4}, \texttt{y0}) manifest as dictionaries mapping the 638 player IDs (strings) to \{0, 1\}, ensuring structural uniformity across gameweeks.

\paragraph{Input: Player Performance Data}

The cleaned dataset (\texttt{cleaned\_data.csv}) provides actual gameweek performance metrics for all players. For each gameweek, the backtester extracts relevant records and merges them with optimizer decisions via \texttt{player\_id}.

\paragraph{Output: Detailed Gameweek Reports}

The backtester produces detailed text reports featuring per-gameweek analyses of points scored, transfer activities, captain selections, automatic substitutions, active lineup configurations, and bench statuses. Concluding summaries aggregate season-long metrics, including total points over 38 gameweeks, cumulative transfers, and average points per gameweek.

\subsubsection{FPL Rule Validation}

The \texttt{fpl\_validator.py} module employs a comprehensive single-pass algorithm to validate all FPL team selection rules.

\paragraph{Validation Algorithm}

The \texttt{fpl\_validator.py} module employs single-pass algorithms validating all FPL 
rules: squad size, positional composition, 
club limits, starting XI size and formation, bench setup, and captaincy.

\paragraph{Formation Validation During Substitutions}

Formation validation during substitutions ensures legality of potentially incomplete lineups 
by checking positional counts, integral to the substitution 
algorithm confirming validity after each prospective bench addition.

\subsubsection{Points Calculation}

The \texttt{fpl\_point\_calculator.py} module implements automatic substitution and captain 
doubling. Following substitutions, the final active lineup yields base points as 
the sum of all active player scores. Captain doubling then applies with fallback: 
if the captain played (non-zero minutes, active), their points add once more; 
otherwise if vice-captain played, their points receive doubling, ensuring precisely 
one bonus.

\subsubsection{Season Orchestration}

The \texttt{backtester.py} module orchestrates season-long processing across 38 gameweeks. 
For each gameweek, it extracts team selections from binary dictionaries, retrieves 
actual performance data, validates selections against FPL rules, and computes 
points via automatic substitutions, tracking cumulative points season-long. 
Transfers are identified via set difference between consecutive gameweek squads: 
bought players appear in current but not previous squad, while sold players appear 
in previous but not current. 

\begin{lstlisting}[language=Python, caption={Transfer Detection Algorithm}]
def detect_transfers(current_squad: List[int], 
                    previous_squad: List[int]) -> Tuple[List, List]:
    """Detect transfers using set difference."""
    current_set = set(current_squad)
    previous_set = set(previous_squad)
    transfers_in = list(current_set - previous_set)
    transfers_out = list(previous_set - current_set)
    return transfers_in, transfers_out
\end{lstlisting}

\paragraph{Output Generation}

Output generation produces detailed formatted reports 
per gameweek (header with points, transfers in/out with player details, captain 
and substitution specifics, active lineup breakdown, bench status), culminating 
in season summaries (total points, transfer count, averages, full-season verification).

\subsubsection{Technical Challenges and Solutions}

The backtester relies on uniform \texttt{player\_id} schemes from data cleaning (638 players) 
to join optimizer decisions with actual performance data from \texttt{cleaned\_data.csv}, 
permitting direct DataFrame filtering and squad extraction. External users must 
adhere to this mapping or fork the repository for custom dataset extensions.

\subsubsection{Implementation Results}

The backtesting phase fulfilled its dual aims of performance evaluation and correctness verification. Detailed per-gameweek reports illuminate optimizer decisions, highlighting patterns in transfer activity, captaincy, and substitutions. These observations drove refinements to the optimization model and affirmed the accuracy of implemented FPL rules against official mechanics. The backtester establishes a basis for comparative evaluations of optimization paradigms and strategies.
