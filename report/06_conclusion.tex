\section{Conclusion and Critical Appraisal}

This project developed and evaluated an optimization framework for Fantasy Premier 
League squad management, leveraging both Mixed Integer Linear Programming and 
Constraint Programming paradigms.

\subsection{Summary of Accomplishments}

The project successfully delivered an end-to-end FPL optimization system spanning 
the complete 2024-25 season (38 gameweeks). Key accomplishments include integrating 
and cleaning 27,222 player-gameweek records from three independent sources, 
reconciling schema inconsistencies and resolving 374 duplicate fixtures; developing 
comprehensive MILP and CP formulations modeling FPL's intricate rules (squad 
composition, budget dynamics with 50\% profit lock, hierarchical starter-bench 
optimization, transfer mechanics with free banking, automatic substitutions); 
achieving successful deployment via PuLP for MILP (CBC, GLPK, SCIP backends) and 
MiniZinc for CP (CP-SAT, Gecode, Chuffed backends); and developing complete 
backtesting infrastructure with dual scenarios—oracle mode yielding approximately 
double human benchmark performance (5,548-5,597 versus 2,810 points for top manager)—
and exhaustive FPL rule validation.

\subsection{Strengths of the Approach}

The project demonstrated notable strengths contributing to its practical value. 
MILP solvers demonstrated superior computational efficiency, resolving each gameweek 
in 0.12-0.24 seconds average, supporting real-time deployment. Technology selections 
emphasize cross-platform accessibility: PuLP furnishes pure-Python MILP environment 
with open-source solvers (CBC, GLPK, SCIP), while MiniZinc supplies well-documented 
constraint modeling with Python integration, both deploying effortlessly across 
Windows, macOS, and Linux without complex compilations or licensing barriers. The 
modular architecture supports independent validation with distinct optimization 
modules for Pre-GW1 Steps 1-2 and Post-GW1, utilities for post-hoc processing, and 
season orchestrator for state management, expediting development through incremental 
validation.

\subsection{Weaknesses and Limitations}

Despite its strengths, the project exhibits significant limitations. The one-week 
forward horizon fundamentally constrains efficacy by addressing each gameweek in 
isolation, overlooking multi-period dynamics including anticipating future price 
escalations, favorable fixture sequences, and chip deployment timing. CP solvers 
demonstrated severe performance limitations rendering them impractical: although 
CP-SAT resolved GW1 optimally in 57 seconds indicating theoretical parity, such 
latencies preclude real-time deployment requiring sub-second responses.

Data integration uncovered substantial quality challenges in open datasets, including 
price and expected points discrepancies across sources and complete prediction 
absences for certain gameweeks. Variability in player participation necessitated 
imputation for standardization, introducing potential distortions. Multiple optimal 
solutions for individual gameweeks engendered unanticipated season-long performance 
variations: diverse bench compositions yield equivalent gameweek scores yet subtly 
alter budget allocation and future transfer options, precipitating butterfly effects 
propagating through subsequent gameweeks.
\subsection{Code and Data Availability}

The complete source code, optimizers, backtesting infrastructure, integrated datasets, 
and comprehensive documentation are publicly available under the MIT License at 
\url{https://github.com/ipsiuhang/FPL-Season-Optimiser}, enabling full reproduction 
and providing a foundation for further research.

\subsection{Future Work and Extensions}

Several directions could extend the current work. Multi-period rolling horizon 
optimization (2-4 weeks) would solve multi-period problems but implement only first 
week's decisions, capturing price rises, fixture runs, and chip timing. Risk-aware 
or non-linear objective functions could incorporate minimal-variance constraints 
modeling manager preferences using MINLP/MIQP solvers. Chip strategy optimization 
could model FPL chips (Wildcard, Bench Boost, Triple Captain, Free Hit) using 
multi-period binary variables constrained by usage limits.