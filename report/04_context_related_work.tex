\section{Related Work}

\subsection{Literature Review}

The literature on fantasy sports optimization has expanded alongside data accessibility and computational advancements,
frequently conceptualizing team selection as constrained optimization problems.

Ramezani and Dinh (2025)~\cite{ramezani2025fpl} develop a data-driven MILP framework for FPL team selection, optimizing the full 15-player squad—including starting 11, bench, and captain—to maximize expected points under budget, position, and club constraints. Their hybrid scoring integrates historical data with linear regression predictions, evaluated on the 2023/24 season via Monte Carlo simulations and ARIMA forecasting, extending to multi-week transfer planning. This informs the current project's MILP formulation, especially squad and captaincy with bench integration. However, it omits automatic substitutions and transfer mechanics—such as free transfer accumulation, four-point penalties, and 50\% profit locks—deferring them to future work. These gaps are filled here via hierarchical pre-GW1 and single-stage post-GW1 models optimizing transfers with lineups, augmented by CP for efficiency comparison.

Becker and Sun (2016)~\cite{becker2016analytical} propose an analytical approach for fantasy football draft and lineup management, utilizing mixed-integer programming for NFL fantasy drafts and weekly lineups to maximize projected wins. Their robust modeling of player projection uncertainties from historical and expert data, combined with positional constraints, informs key design choices in the current work. Specifically, their treatment of uncertainty without explicit stochastic variables encouraged the adoption of conditional expected scores as inputs, accounting for factors like player availability. Additionally, their hierarchical approach to lineup optimization—applied in the draft phase to ensure positional depth for starters plus injury buffers before weekly starter selections—inspired similar prioritization in this project's formulation, adapted to FPL's unique rules such as club limits and substitution mechanics.

Groos (2025)~\cite{groos2025openfpl} introduces OpenFPL, an open-source forecasting method that rivals state-of-the-art Fantasy Premier League services. This ensemble model combines XGBoost and Random Forest, trained on public data from the 2020-21 to 2023-24 seasons, with prospective evaluation on 2024-25. It forecasts points over one- to three-gameweek horizons, excelling in high-return player predictions. This work exemplifies the prediction-focused research strand in FPL analytics, which complements the current project's optimization focus. By treating predictions as external inputs, this project concentrates on the mechanics of team selection, transfers, and substitutions, making the work more accessible as an educational resource and allowing practitioners to integrate predictions from various sources, including statistical models like OpenFPL or expert judgment.
\subsection{Data Provision and Infrastructure}

The FPL-Elo-Insights repository~\cite{olbauday2025fpl} provides comprehensive 
datasets for the 2024-25 FPL season, integrating official API data with match 
statistics and team ratings. Organized by gameweek with standardized player IDs, 
it served as a primary data source for this study. 

The Fantasy-Premier-League repository~\cite{vaastav2025fpl} offers CSV datasets 
with player statistics and gameweek-specific performance data for the 2024-25 
season, supporting historical analysis and validation against actual outcomes.

The fplcache repository~\cite{randdalf2024fplcache} maintains daily caches of 
official FPL API data, aiding in understanding API structures for data 
integration efforts.

